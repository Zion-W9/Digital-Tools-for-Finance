\documentclass[../main.tex]{subfiles}
\begin{document}

\section{Literatures}

A theoretical relationship between stock prices and productivity within a country is derived by Kung and Schmid (2015)\cite{Kung2015}. They construct a general equilibrium stochastic growth model with endogenous productivity growth and asset prices. The model shows that favorable economic conditions boost innovation and the development of new technologies. Since technological progress fosters long-run economic growth, endogenous innovation generates a powerful propagation mechanism for shocks reflected in persistent variation in long-term growth prospects.

Empirically, there is evidence that stock returns are strongly related (though opposite signs) to measures of future real activity\cite{Fama1981}. This is consistent with a rational expectations view in which markets for goods and securities set current prices on the basis of forecasts of relevant real variables. in 17 advanced economics between 1870 and 2016, the data reveals two broad eras of stock market growth: the stock market grew at the same rate long-run rate as GDP during the period before the 1980s; and the capitalization growth accelerated far beyond that in GDP after the 1980s\cite{KUVSHINOV2022}. From 1991 to 2012, changes in share prices provide reliable predictions of near term future economic growth in the USA and the UK. However, changes in economic growth are not related to share price movements, while in the case of Japan, share price movements do not appear to be a useful leading predictor for near term economic growth and vice-versa\cite{Hossain2015}.

This paper studies the long run equilibrium relationship between stock prices and domestic GDP for the OECD countries. We expect to find higher stock price growth in countries high GDP growth than in countries with low GDP growth.


\end{document}
